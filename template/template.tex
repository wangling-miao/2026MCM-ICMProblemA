%% MCM/ICM Paper Template
%% 
\documentclass{mcmthesis}
\mcmsetup{CTeX = false,   % 使用英文宏包
          tcn = {1234567}, % 你的队伍控制号
          problem = A,     % 选题 (A/B/C/D/E/F)
          sheet = true,    % 是否输出摘要页
          titleinsheet = true, % 摘要页是否包含标题
          keywordsinsheet = true, % 摘要页是否包含关键词
          titlepage = false, 
          abstract = true}

\usepackage{newtxtext}     % Times New Roman 字体
\usepackage{amsmath, amssymb, amsthm}
\usepackage{geometry}
\setlength{\headheight}{14pt} 
\usepackage{graphicx}
\usepackage{float}
\usepackage{booktabs}      % 三线表
\usepackage{tabularx}      % 自动宽度的表格
\usepackage{longtable}     % 长表格
\usepackage{hyperref}      % 超链接
\usepackage{xcolor}
\usepackage{listings}      % 代码块
\usepackage{lipsum}  

% 标题设置
\title{Your Paper Title Here}

\begin{document}

% =========================================
% 0. 摘要页 (Summary Sheet)
% =========================================
\begin{abstract}
    % [第一段:背景与核心目标]
    % 引用使用如下格式
    \cite{chen2020global}

    % [Problem 1 摘要]
    For \textbf{Problem 1}, 

    % [Problem 2 摘要]
    For \textbf{Problem 2},

    % [Problem 3 摘要]
    For \textbf{Problem 3},

    % [Problem 4 摘要]
    For \textbf{Problem 4}, 

    % [结论]
    In conclusion, 

    \begin{keywords}
    Keyword 1; Keyword 2; Keyword 3; Algorithm Name; Model Name
    \end{keywords}
\end{abstract}

\maketitle

% =========================================
% 目录
% =========================================
\tableofcontents
\newpage

% =========================================
% 1. Introduction (引言)
% =========================================
\section{Introduction}

\subsection{Background}

\subsection{Problem Restatement}

\subsection{Literature Review}

\subsection{Our Work}

% =========================================
% 2. Assumptions and Notations (假设与符号)
% =========================================
\section{Model Assumptions and Notations}

\subsection{Model Assumptions}

\subsection{Notations}

% =========================================
% 3. Data Pre-processing (数据预处理)
% =========================================
\section{Data Pre-processing}


\section{Model Development}
% =========================================
% 4. Model for Problem 1 (问题1建模)
% =========================================
\subsection{Model for Problem 1:}

% =========================================
% 5. Model for Problem 2 (问题2建模)
% =========================================
\subsection{Model for Problem 2:}

% =========================================
% 6. Model for Problem 3 (问题3建模)
% =========================================
\subsection{Model for Problem 3:}

% =========================================
% 7. Model for Problem 4 (问题4建模)
% =========================================
\subsection{Model for Problem 4:}

% =========================================
% 8. Sensitivity Analysis (灵敏度分析)
% =========================================
\section{Sensitivity Analysis}

% =========================================
% 10. Model Extension (模型推广)
% =========================================
\section{Model Extension}

% =========================================
% 9. Conclusion (结论)
% =========================================
\section{Conclusion}

% =========================================
% References (参考文献)
% =========================================
\bibliographystyle{unsrt} 
\bibliography{ref}

% =========================================
% Appendices (附录)
% =========================================
\newpage
\appendix

% ----------------------------------------------------
% 关键修改:将目录深度设为0,禁止Section级别出现在目录中
% ----------------------------------------------------
\addtocontents{toc}{\protect\setcounter{tocdepth}{0}}

\section{Report on Use of AI}
% 必须包含的 AI 使用报告部分
\begin{itemize}
    \item \textbf{OpenAI ChatGPT (GPT-4o)}
    \begin{itemize}
        \item \textbf{Query:} How to implement Entropy Weight Method in Python?
        \item \textbf{Usage:} The code generated was used in Section 5.1 for weight calculation.
    \end{itemize}
    
    \item \textbf{Perplexity AI}
    \begin{itemize}
        \item \textbf{Query:} Latest statistics on global AI supercomputing power 2025.
        \item \textbf{Usage:} Data retrieved was used in Section 3.1.
    \end{itemize}
\end{itemize}

\label{LastPage}
\end{document}